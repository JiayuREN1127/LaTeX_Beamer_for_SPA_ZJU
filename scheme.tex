\documentclass[aspectratio=169]{beamer}

\usepackage[nofonts]{ctex}
\usepackage{xeCJK}
\usepackage{graphicx}
\usepackage{fontawesome5}
\usepackage{xcolor}
\usepackage{amsmath}
\usepackage{tikz}
\usepackage{bookmark}
\usepackage{hyperref}
\usepackage{microtype}
\usepackage{fontspec}
\usepackage{setspace}
\usepackage{etoolbox}
\usepackage{mathptmx}
\usepackage{bm}
\usepackage{booktabs}
\usepackage[
    backend=bibtex,
    style=gb7714-2015,      % 使用 GB/T 7714-2015 样式
    gbbiblabel=bracket,     % 文献编号用方括号
    sorting=none,     
]{biblatex}
% \addbibresource{references.bib}
\usetikzlibrary{intersections}
\usetikzlibrary{positioning, shapes.geometric}
\usetikzlibrary{calc}
\usetikzlibrary{shadows}



% 主题设置
\usecolortheme{default} 
\definecolor{zjublue}{RGB}{0,63,136}
\definecolor{zjured}{RGB}{176,31,36}
\definecolor{text}{RGB}{231,76,60}

% 字体设置
\setCJKfamilyfont{BTS}{HGBTS_CNKI}[
    Path = ./fonts/,
    Extension = .ttf
]
\setCJKfamilyfont{XBS}{HGXBS_CNKI}[
    Path = ./fonts/,
    Extension = .ttf
]
\setCJKmainfont{simsun}[
    Path = ./fonts/,
    Extension = .ttc,
    BoldFont = simsun,
    AutoFakeBold = 2
]
\setCJKfamilyfont{kai}{simkai}[
    Path = ./fonts/,
    Extension = .ttf,
    BoldFont = simkai,
    AutoFakeBold = 2
]
\setmainfont{Times New Roman}[
    BoldFont = {Times New Roman Bold}
]
\makeatletter
\def\tagform@#1{\maketag@@@{{\rmfamily\fontsize{6pt}{6pt}\selectfont(}\rmfamily\fontsize{6pt}{6pt}\selectfont#1{\rmfamily\fontsize{6pt}{6pt}\selectfont)}}}
\makeatother


\begin{document}


% 封面
\begin{frame}[plain]
    
    % 封面底色形状
    \begin{tikzpicture}[remember picture,overlay]
        
        \pgfmathsetlengthmacro{\rectheight}{\paperheight * 7 / 10}
        
        \fill[zjured] 
            (current page.north west) -- 
            (current page.north east) --
            ($(current page.north east)-(0,\rectheight)$) --
            ($(current page.north west)-(0,\rectheight)$) -- cycle;
        
        \fill[zjublue] 
            (current page.north west) -- 
            ($(current page.north east)-(0,\rectheight)$) --
            ($(current page.north east)-(0,\rectheight)$) --
            (current page.north east) -- cycle;

        \node[anchor=north west, inner sep=0] at 
            ($(current page.north west)+(0.5cm,-0.3cm)$) 
            {\includegraphics[height=1.2cm]{images/logo.png}};
       
        \begin{scope}[shift={($(current page.south west)+(0.5cm,1.75cm)$)}, scale=0.8, yscale=-1]
            \path[fill=none, draw=none, name path=leftL]
            (0,0) -- ++(0,1.5cm) -- ++(1.5cm,0) -- ++(0,-0.5cm) -- ++(-1cm,0) -- ++(0,-1cm) -- cycle;
            \path[left color=zjublue, right color=zjured, shading angle=-135, opacity=0.9]
            (0,0) -- ++(0,1.5cm) -- ++(1.5cm,0) -- ++(0,-0.5cm) -- ++(-1cm,0) -- ++(0,-1cm) -- cycle;
        \end{scope}

        \begin{scope}[shift={($(current page.south east)+(-0.5cm,1.75cm)$)}, scale=0.8]
            \path[fill=none, draw=none, name path=rightL]
                (0,0) -- ++(0,-1.5cm) -- ++(-1.5cm,0) -- ++(0,0.5cm) -- ++(1cm,0) -- ++(0,1cm) -- cycle;
            \path[left color=zjured, right color=zjublue, shading angle=135, opacity=0.9]
                (0,0) -- ++(0,-1.5cm) -- ++(-1.5cm,0) -- ++(0,0.5cm) -- ++(1cm,0) -- ++(0,1cm) -- cycle;
        \end{scope}


    \end{tikzpicture}

    % 封面文本
    \begin{tikzpicture}[remember picture, overlay]
       
        % 活动名称
        \node (activity_shadow) at ([xshift=1pt,yshift=-1pt,yshift=2.3cm]current page.center) {\rmfamily\CJKfamily{simsun} \color{black!90}
        活动名称:请输入文本
        };
        \node (activity) at ([yshift=2.3cm]current page.center) {\rmfamily\CJKfamily{simsun} \color{white} 
        活动名称:请输入文本
        };
        \draw[white, thin] (activity.west) -- (current page.west |- activity.west);
        \draw[white, thin] (activity.east) -- (current page.east |- activity.east);

        % 标题
        \node at ([xshift=2pt,yshift=-2pt,yshift=0.6cm]current page.center) {\rmfamily\CJKfamily{BTS} \color{black!90} \fontsize{40pt}{5pt}\selectfont
        标题:请输入文本
        };
        \node at ([yshift=0.6cm]current page.center) {\rmfamily\CJKfamily{BTS} \color{white} \fontsize{40pt}{5pt}\selectfont
        标题:请输入文本
        };
        \node at ([yshift=-0.6cm]current page.center) {\rmfamily \color{white} \large 
        Subtitle: Please enter the text.
        };

        % 指导教师
        \node at ([yshift=-2.3cm]current page.center) {\rmfamily\CJKfamily{kai} \fontsize{10pt}{5pt}\selectfont
        指导教师:请输入文本
        };
        % 所属单位
        \node at ([yshift=-2.8cm]current page.center) {\rmfamily\CJKfamily{kai} \fontsize{10pt}{5pt}\selectfont
        所属单位:请输入文本
        };
        % 姓名
        \node at ([yshift=-3.3cm]current page.center) {\rmfamily\CJKfamily{kai} \fontsize{10pt}{5pt}\selectfont
        姓名:请输入文本
        };
        % 日期
        \node at ([yshift=-3.8cm]current page.center) {\rmfamily\CJKfamily{kai} \fontsize{10pt}{5pt}\selectfont
        日期:\today
        };
    
    \end{tikzpicture}

\end{frame}


% 目录
\begin{frame}
    
    \begin{tikzpicture}[remember picture, overlay]

        % 渐变矩形背景
        \path[rounded corners=0pt, left color=zjublue, right color=zjured, shading angle=-45, opacity=1]
        ([xshift=-1.3cm,yshift=-2cm]current page.north) rectangle ([xshift=1.3cm,yshift=-1.0cm]current page.north);

        % 目录文本
        \node at ([xshift=1pt,yshift=-1.5cm,yshift=-1pt]current page.north) {\rmfamily\CJKfamily{BTS} \fontsize{18pt}{5pt}\selectfont \color{black}
        \kern0.2em 
        目
        \kern1em 
        录
        \kern0.2em
        };
        \node at ([yshift=-1.5cm]current page.north) {\rmfamily\CJKfamily{BTS} \fontsize{18pt}{5pt}\selectfont \color{white}
        \kern0.2em 
        目
        \kern1em 
        录
        \kern0.2em
        };

        \node[anchor=north east, inner sep=0] at 
        ($(current page.north east)+(-0.5cm,-0.3cm)$) 
        {\includegraphics[height=1.2cm]{images/logo_color.png}};

        \node at ([xshift=1pt,yshift=1cm]current page.south) {\rmfamily\CJKfamily{BTS} \fontsize{9pt}{0pt}\selectfont \color{black!80}
        标题:请输入文本
        };
        \node at ([xshift=1pt,yshift=0.6cm]current page.south) {\rmfamily\CJKfamily{BTS} \fontsize{8pt}{0pt}\selectfont \color{black!80}
        Subtitle: Please enter the text.
        };


        \end{tikzpicture}

    \vspace{0.7cm}
    \centering
    {\CJKfamily{XBS} \color{black} 
    \setbeamercolor{structure}{fg=black}
    \fontsize{12pt}{10pt}\selectfont
    \setbeamertemplate{section in toc}[sections numbered]
    \setbeamertemplate{section in toc}{
    {\rmfamily\inserttocsectionnumber.}~\inserttocsection\par\vspace{-0.6em}}
    \tableofcontents[hideallsubsections]}
    

\end{frame}



% 正文

% 第一节
\section{请输入文本(背景与意义)}

    \begin{frame}{请输入文本(背景与意义)}

        %页面底色
        \begin{tikzpicture}[remember picture,overlay]
            
            \pgfmathsetlengthmacro{\rectheight}{\paperheight * 7 / 10}
                    
                    \fill[zjured] 
                        (current page.north west) -- 
                        (current page.north east) --
                        ($(current page.north east)-(0,\rectheight)$) --
                        ($(current page.north west)-(0,\rectheight)$) -- cycle;
                                        
                    \node[anchor=north east, inner sep=0] at 
                        ($(current page.north east)+(-0.5cm,-0.3cm)$) 
                        {\includegraphics[height=1.2cm]{images/logo.png}};

                    \begin{scope}[shift={($(current page.south west)+(0.5cm,0.5cm)$)}]
                        \path[rounded corners=0pt, left color=zjublue, right color=zjublue]
                            (0.7cm,0) rectangle (2cm,0.5cm);
                        \path[rounded corners=0pt, left color=zjublue, right color=zjured, shading angle=-135]
                            (0,0) rectangle (0.5cm,0.5cm);            
                    \end{scope}

                    \begin{scope}[shift={($(current page.south east)+(-0.5cm,0.5cm)$)}, xscale=-1]
                        \path[rounded corners=0pt, left color=zjublue, right color=zjublue]
                            (0.7cm,0) rectangle (2cm,0.5cm);
                        \path[rounded corners=0pt, left color=zjublue, right color=zjured, shading angle=135]
                            (0,0) rectangle (0.5cm,0.5cm);            
                    \end{scope}



        \end{tikzpicture}

        %页面文本
        \begin{tikzpicture}[remember picture, overlay]
        
            % 项目名称
            \node (activity_shadow) at ([xshift=1pt,yshift=-1pt,yshift=2.3cm]current page.center) {\centering\rmfamily\CJKfamily{simsun} \color{black!90}
            项目名称:请输入文本
            };
            \node (activity) at ([yshift=2.3cm]current page.center) {\centering\rmfamily\CJKfamily{simsun} \color{white} 
            项目名称:请输入文本
            };
            \draw[white, thin] (activity.west) -- (current page.west |- activity.west);
            \draw[white, thin] (activity.east) -- (current page.east |- activity.east);

            % 标题
            \node at ([xshift=2pt,yshift=-2pt,yshift=0.6cm]current page.center) {\centering\rmfamily\CJKfamily{BTS} \color{black!90} \fontsize{30pt}{5pt}\selectfont
            一、请输入文本(背景与意义)
            };
            \node at ([yshift=0.6cm]current page.center) {\centering\rmfamily\CJKfamily{BTS} \color{white} \fontsize{30pt}{5pt}\selectfont
            一、请输入文本(背景与意义)
            };
            \node at ([yshift=-0.7cm]current page.center) {\centering\rmfamily \color{white} \large 
            Subtitle: Please enter the text.
            };

            \node [text width=20cm] at ([yshift=-3cm]current page.center) {\centering\rmfamily\CJKfamily{kai} \fontsize{10pt}{15pt}\selectfont \setlength{\linespread{0.9}}
            小节内容1:请输入文本\\
            小节内容2:请输入\\
            小节内容3:文本Please\\
            };
            
        \end{tikzpicture}

    \end{frame}

    % 页面一
    \begin{frame}
        
        \begin{tikzpicture}[remember picture, overlay]

            \node[anchor=north west, inner sep=0pt] (boxshadow) at ([xshift=0.5cm,yshift=-0.5cm]current page.north west) {
                
                \begin{tikzpicture}                
                \node[fill=zjublue, text width=3cm, minimum height=0.8cm, 
                align=center,
                font=\rmfamily\CJKfamily{BTS}\fontsize{14pt}{14pt}\selectfont] 
                at (0,0) {
                    \begin{tikzpicture}[baseline=(textnode.base)]
                    \node[text=black!90, opacity=0.5, shift={(0.8pt,-0.5pt)}] (shadow) at (0,0) {\addfontfeature{LetterSpace=200}
                    研究意义};
                    \node[text=white] (textnode) at (0,0) {\addfontfeature{LetterSpace=200}
                    研究意义};
                    \end{tikzpicture}
                };
                \end{tikzpicture}

            };

            \draw[fill=zjublue, draw=none]
                ([xshift=0.1cm]boxshadow.north east) rectangle
                ([xshift=\dimexpr\paperwidth-0.5cm\relax,yshift=-0.8cm]current page.north west);

            \draw[fill=zjured, draw=none]
                ([xshift=0.1cm]boxshadow.south east) rectangle
                ([xshift=\dimexpr\paperwidth-0.5cm\relax,yshift=-0.88cm]current page.north west);

            \node[anchor=west, align=left, text=white, font=\rmfamily\CJKfamily{XBS}\fontsize{7pt}{16pt}\selectfont]
            at ([xshift=0.1cm,yshift=-0.6cm]boxshadow.north east)
            {\spaceskip=0.5em plus 0.2em minus 0.1em 研究目的与现实意义};

            \node[anchor=south east, inner sep=0] at ([xshift=-0.2cm,yshift=0.3cm]current page.south east) 
                {\includegraphics[height=1.3cm]{images/logo_square.png}};

        \end{tikzpicture}
            
        \begin{tikzpicture}[remember picture, overlay]
            
            \node[anchor=north west, text width=15cm, align=left, font=\rmfamily\CJKfamily{XBS}\fontsize{10pt}{10pt}\selectfont] at ([xshift=0.2cm,yshift=-1.2cm]current page.north west) 
            {
                \setbeamertemplate{itemize item}{\color{black}$\bullet$}
                \setbeamertemplate{itemize subitem}{\color{black}$\circ$}

                \begin{itemize}
                
                    \item \textbf{标题一}
                    \begin{itemize}
                        \item {\rmfamily\CJKfamily{simsun}\fontsize{8pt}{8pt} 这是标题一的正文,采用宋体书写,内容可以根据实际需要进行扩展和修改。} \\                    
                        \item {\rmfamily\CJKfamily{kai}\fontsize{8pt}{8pt} 这里是标题一的补充说明,采用楷体书写,进一步阐述相关内容。} \\
                        [0.8em]
                    \end{itemize}
                    
                    \item \textbf{标题二} \\[0.3em]
                    {\rmfamily\CJKfamily{simsun}\fontsize{8pt}{8pt} 这是标题二的正文,采用宋体书写,内容可以根据实际需要进行扩展和修改。内容可以根据实际需要进行扩展和修改。内容可以根据实际需要进行扩展和修改。内容可以根据实际需要进行扩展和修改。} \\                    
                    {\rmfamily\CJKfamily{kai}\fontsize{8pt}{8pt} 这里是标题二的补充说明,采用楷体书写,进一步阐述相关内容。} \\
                    [0.8em]

                    \item \textbf{标题三} \\[0.3em]
                    {\rmfamily\CJKfamily{simsun}\fontsize{8pt}{8pt} 这是标题三的正文,采用宋体书写,内容可以根据实际需要进行扩展和修改。} \\                    
                    {\rmfamily\CJKfamily{kai}\fontsize{8pt}{8pt} 这里是标题三的补充说明,采用楷体书写,进一步阐述相关内容。} \\
                    [0.8em]

                    \item \textbf{标题四} \\[0.3em]
                    {\rmfamily\CJKfamily{simsun}\fontsize{8pt}{8pt} 这是标题四的正文,采用宋体书写,内容可以根据实际需要进行扩展和修改。} \\                    
                    {\rmfamily\CJKfamily{kai}\fontsize{8pt}{8pt} 这里是标题四的补充说明,采用楷体书写,进一步阐述相关内容。} \\
                    [0.8em]

                    \item \textbf{标题五} \\[0.3em]
                    {\rmfamily\CJKfamily{simsun}\fontsize{8pt}{8pt} 这是标题五的正文,采用宋体书写,内容可以根据实际需要进行扩展和修改。} \\                    
                    [0.8em]
                
                \end{itemize}

            };

        \end{tikzpicture}

    \end{frame}


% 第二节
\section{请输入文本(内容与思路)}

    \begin{frame}{请输入文本(内容与思路)}

        %页面底色
        \begin{tikzpicture}[remember picture,overlay]
            
            \pgfmathsetlengthmacro{\rectheight}{\paperheight * 7 / 10}
                    
                    \fill[zjured] 
                        (current page.north west) -- 
                        (current page.north east) --
                        ($(current page.north east)-(0,\rectheight)$) --
                        ($(current page.north west)-(0,\rectheight)$) -- cycle;
                                        
                    \node[anchor=north east, inner sep=0] at 
                        ($(current page.north east)+(-0.5cm,-0.3cm)$) 
                        {\includegraphics[height=1.2cm]{images/logo.png}};

                    \begin{scope}[shift={($(current page.south west)+(0.5cm,0.5cm)$)}]
                        \path[rounded corners=0pt, left color=zjublue, right color=zjublue]
                            (0.7cm,0) rectangle (2cm,0.5cm);
                        \path[rounded corners=0pt, left color=zjublue, right color=zjured, shading angle=-135]
                            (0,0) rectangle (0.5cm,0.5cm);            
                    \end{scope}

                    \begin{scope}[shift={($(current page.south east)+(-0.5cm,0.5cm)$)}, xscale=-1]
                        \path[rounded corners=0pt, left color=zjublue, right color=zjublue]
                            (0.7cm,0) rectangle (2cm,0.5cm);
                        \path[rounded corners=0pt, left color=zjublue, right color=zjured, shading angle=135]
                            (0,0) rectangle (0.5cm,0.5cm);            
                    \end{scope}



        \end{tikzpicture}

        %页面文本
        \begin{tikzpicture}[remember picture, overlay]
        
            % 项目名称
            \node (activity_shadow) at ([xshift=1pt,yshift=-1pt,yshift=2.3cm]current page.center) {\centering\rmfamily\CJKfamily{simsun} \color{black!90}
            项目名称:请输入文本
            };
            \node (activity) at ([yshift=2.3cm]current page.center) {\centering\rmfamily\CJKfamily{simsun} \color{white} 
            项目名称:请输入文本
            };
            \draw[white, thin] (activity.west) -- (current page.west |- activity.west);
            \draw[white, thin] (activity.east) -- (current page.east |- activity.east);

            % 标题
            \node at ([xshift=2pt,yshift=-2pt,yshift=0.6cm]current page.center) {\centering\rmfamily\CJKfamily{BTS} \color{black!90} \fontsize{30pt}{5pt}\selectfont
            二、请输入文本(内容与思路)
            };
            \node at ([yshift=0.6cm]current page.center) {\centering\rmfamily\CJKfamily{BTS} \color{white} \fontsize{30pt}{5pt}\selectfont
            二、请输入文本(内容与思路)
            };
            \node at ([yshift=-0.7cm]current page.center) {\centering\rmfamily \color{white} \large 
            Subtitle: Please enter the text.
            };

            \node [text width=20cm] at ([yshift=-3cm]current page.center) {\centering\rmfamily\CJKfamily{kai} \fontsize{10pt}{15pt}\selectfont \setlength{\linespread{0.9}}
            小节内容1:请输入文本\\
            小节内容2:请输入\\
            小节内容3:文本Please\\
            };
            
        \end{tikzpicture}

    \end{frame}

    % 页面二
    \begin{frame}
        
        \begin{tikzpicture}[remember picture, overlay]

            \node[anchor=north west, inner sep=0pt] (boxshadow) at ([xshift=0.5cm,yshift=-0.5cm]current page.north west) {
                
                \begin{tikzpicture}                
                \node[fill=zjublue, text width=3cm, minimum height=0.8cm, 
                align=center,
                font=\rmfamily\CJKfamily{BTS}\fontsize{14pt}{14pt}\selectfont] 
                at (0,0) {
                    \begin{tikzpicture}[baseline=(textnode.base)]
                    \node[text=black!90, opacity=0.5, shift={(0.8pt,-0.5pt)}] (shadow) at (0,0) {\addfontfeature{LetterSpace=200}
                    研究意义};
                    \node[text=white] (textnode) at (0,0) {\addfontfeature{LetterSpace=200}
                    研究意义};
                    \end{tikzpicture}
                };
                \end{tikzpicture}

            };

            \draw[fill=zjublue, draw=none]
                ([xshift=0.1cm]boxshadow.north east) rectangle
                ([xshift=\dimexpr\paperwidth-0.5cm\relax,yshift=-0.8cm]current page.north west);

            \draw[fill=zjured, draw=none]
                ([xshift=0.1cm]boxshadow.south east) rectangle
                ([xshift=\dimexpr\paperwidth-0.5cm\relax,yshift=-0.88cm]current page.north west);

            \node[anchor=west, align=left, text=white, font=\rmfamily\CJKfamily{XBS}\fontsize{7pt}{16pt}\selectfont]
            at ([xshift=0.1cm,yshift=-0.6cm]boxshadow.north east)
            {\spaceskip=0.5em plus 0.2em minus 0.1em 研究目的与现实意义};

            \path[fill=none, left color=zjublue, right color=zjured, shading angle=45, opacity=1, draw=none]
                ([xshift=0cm,yshift=1.2cm]current page.south west) rectangle ([xshift=0cm,yshift=1.22cm]current page.south east);

            \node[anchor=south east, inner sep=0] at ([xshift=-0.2cm,yshift=0.3cm]current page.south east) 
                {\includegraphics[height=1.3cm]{images/logo_square.png}};

        \end{tikzpicture}
            
        \begin{tikzpicture}[remember picture, overlay]
            
            \node[anchor=north west, text width=10.2cm, align=justify, font=\rmfamily\CJKfamily{XBS}\fontsize{10pt}{10pt}\selectfont] at ([xshift=0.2cm,yshift=-1.2cm]current page.north west) 
            {
                \setbeamertemplate{itemize item}{\color{black}$\bullet$}
                \setbeamertemplate{itemize subitem}{\color{black}$\circ$}

                \begin{itemize}
                
                    \item \textbf{标题一}
                    \begin{itemize}
                        \item {\rmfamily\CJKfamily{simsun}\fontsize{8pt}{8pt} 这是标题一的正文,采用宋体书写。} \\
                        \item {\rmfamily\CJKfamily{kai}\fontsize{8pt}{8pt} 这里是标题一的补充说明,采用楷体书写,进一步阐述相关内容。} \\
                        [0.8em]
                    \end{itemize}
                    
                    \item \textbf{标题二} \\[0.3em]
                    {\rmfamily\CJKfamily{simsun}\fontsize{8pt}{8pt} 这是标题二的正文,采用宋体书写,内容可以根据实际需要进行扩展和修改。内容可以根据实际需要进行扩展和修改。内容可以根据实际需要进行扩展和修改。内容可以根据实际需要进行扩展和修改。} \\                    
                    {\rmfamily\CJKfamily{kai}\fontsize{8pt}{8pt} 这里是标题二的补充说明,采用楷体书写,进一步阐述相关内容。} \\
                    [0.8em]

                    \item \textbf{标题三} \\[0.3em]
                    {\rmfamily\CJKfamily{simsun}\fontsize{8pt}{8pt} 这是标题三的正文,采用宋体书写,内容可以根据实际需要进行扩展和修改。} \\
                    {\rmfamily\CJKfamily{kai}\fontsize{8pt}{8pt} 这里是标题三的补充说明,采用楷体书写,进一步阐述相关内容。} \\
                    [0.8em]

                    \item \textbf{标题四} \\[0.3em]
                    {\rmfamily\CJKfamily{simsun}\fontsize{8pt}{8pt} 这是标题四的正文,采用宋体书写,内容可以根据实际需要进行扩展和修改。} \\
                    [0.8em]

                            
                \end{itemize}

            };

            \node[anchor=north east, text width=4cm, align=justify, font=\rmfamily\CJKfamily{XBS}\fontsize{10pt}{5pt}\selectfont] 
                at ([xshift=-0.5cm,yshift=-1.8cm]current page.north east) 
            {
                
               {\centering
                    \begin{tikzpicture}
                        \node[text width=4cm, align=center]{\color{zjured}这是\color{zjublue}右侧文本框标题};
                    \end{tikzpicture}\\[1em]
                }
                
                {\rmfamily\CJKfamily{simsun}\fontsize{8pt}{8pt} 这是右侧文本框的正文,采用宋体书写,内容可以根据实际需要进行扩展和修改。} \\
                {\rmfamily\CJKfamily{kai}\fontsize{8pt}{8pt} 这里是右侧文本框的补充说明,采用楷体书写,进一步阐述相关内容。} \\
        
            };

            \node[anchor=north east, text width=4.3cm, align=justify, font=\rmfamily\CJKfamily{XBS}\fontsize{10pt}{5pt}\selectfont] 
                at ([xshift=-0.5cm,yshift=-4.5cm]current page.north east) 
            {
                            
                \setbeamertemplate{itemize item}{\color{black}$\circ$}
                \begin{itemize}
                    \item {\rmfamily\CJKfamily{simsun}\fontsize{8pt}{8pt} 这是右侧文本框的正文,采用宋体书写,内容可以根据实际需要进行扩展和修改。} \\
                    \item {\rmfamily\CJKfamily{kai}\fontsize{8pt}{8pt} 这里是右侧文本框的补充说明,采用楷体书写,进一步阐述相关内容。} \\
                \end{itemize}
            
            };


            \node[anchor=north west, text width=10.2cm, align=justify, font=\rmfamily\CJKfamily{kai}\fontsize{6pt}{6pt}\selectfont, fill=none, draw=none] 
                at ([xshift=0.2cm,yshift=1.2cm]current page.south west) 
            {
                这里是放置文献或脚注的地方,可根据需要填写。\\
                This is an additional text box for literature or footnotes aligned with the one above. \\
                这里是放置文献或脚注的地方,可根据需要填写。\\
                This is an additional text box for literature or footnotes aligned with the one above.
            };
            

        \end{tikzpicture}

    \end{frame}


% 第三节
\section{请输入文本(研究设计)}

    \begin{frame}{请输入文本(研究设计)}

        %页面底色
        \begin{tikzpicture}[remember picture,overlay]
            
            \pgfmathsetlengthmacro{\rectheight}{\paperheight * 7 / 10}
                    
                    \fill[zjured] 
                        (current page.north west) -- 
                        (current page.north east) --
                        ($(current page.north east)-(0,\rectheight)$) --
                        ($(current page.north west)-(0,\rectheight)$) -- cycle;
                                        
                    \node[anchor=north east, inner sep=0] at 
                        ($(current page.north east)+(-0.5cm,-0.3cm)$) 
                        {\includegraphics[height=1.2cm]{images/logo.png}};

                    \begin{scope}[shift={($(current page.south west)+(0.5cm,0.5cm)$)}]
                        \path[rounded corners=0pt, left color=zjublue, right color=zjublue]
                            (0.7cm,0) rectangle (2cm,0.5cm);
                        \path[rounded corners=0pt, left color=zjublue, right color=zjured, shading angle=-135]
                            (0,0) rectangle (0.5cm,0.5cm);            
                    \end{scope}

                    \begin{scope}[shift={($(current page.south east)+(-0.5cm,0.5cm)$)}, xscale=-1]
                        \path[rounded corners=0pt, left color=zjublue, right color=zjublue]
                            (0.7cm,0) rectangle (2cm,0.5cm);
                        \path[rounded corners=0pt, left color=zjublue, right color=zjured, shading angle=135]
                            (0,0) rectangle (0.5cm,0.5cm);            
                    \end{scope}



        \end{tikzpicture}

        %页面文本
        \begin{tikzpicture}[remember picture, overlay]
        
            % 项目名称
            \node (activity_shadow) at ([xshift=1pt,yshift=-1pt,yshift=2.3cm]current page.center) {\centering\rmfamily\CJKfamily{simsun} \color{black!90}
            项目名称:请输入文本
            };
            \node (activity) at ([yshift=2.3cm]current page.center) {\centering\rmfamily\CJKfamily{simsun} \color{white} 
            项目名称:请输入文本
            };
            \draw[white, thin] (activity.west) -- (current page.west |- activity.west);
            \draw[white, thin] (activity.east) -- (current page.east |- activity.east);

            % 标题
            \node at ([xshift=2pt,yshift=-2pt,yshift=0.6cm]current page.center) {\centering\rmfamily\CJKfamily{BTS} \color{black!90} \fontsize{30pt}{5pt}\selectfont
            三、请输入文本(研究设计)
            };
            \node at ([yshift=0.6cm]current page.center) {\centering\rmfamily\CJKfamily{BTS} \color{white} \fontsize{30pt}{5pt}\selectfont
            三、请输入文本(研究设计)
            };
            \node at ([yshift=-0.7cm]current page.center) {\centering\rmfamily \color{white} \large 
            Subtitle: Please enter the text.
            };

            \node [text width=20cm] at ([yshift=-3cm]current page.center) {\centering\rmfamily\CJKfamily{kai} \fontsize{10pt}{15pt}\selectfont \setlength{\linespread{0.9}}
            小节内容1:请输入文本\\
            小节内容2:请输入\\
            小节内容3:文本Please\\
            };
            
        \end{tikzpicture}

    \end{frame}

    % 页面三
    \begin{frame}
        
        \begin{tikzpicture}[remember picture, overlay]

            \node[anchor=north west, inner sep=0pt] (boxshadow) at ([xshift=0.5cm,yshift=-0.5cm]current page.north west) {
                
                \begin{tikzpicture}                
                \node[fill=zjublue, text width=3cm, minimum height=0.8cm, 
                align=center,
                font=\rmfamily\CJKfamily{BTS}\fontsize{14pt}{14pt}\selectfont] 
                at (0,0) {
                    \begin{tikzpicture}[baseline=(textnode.base)]
                    \node[text=black!90, opacity=0.5, shift={(0.8pt,-0.5pt)}] (shadow) at (0,0) {\addfontfeature{LetterSpace=200}
                    研究意义};
                    \node[text=white] (textnode) at (0,0) {\addfontfeature{LetterSpace=200}
                    研究意义};
                    \end{tikzpicture}
                };
                \end{tikzpicture}

            };

            \draw[fill=zjublue, draw=none]
                ([xshift=0.1cm]boxshadow.north east) rectangle
                ([xshift=\dimexpr\paperwidth-0.5cm\relax,yshift=-0.8cm]current page.north west);

            \draw[fill=zjured, draw=none]
                ([xshift=0.1cm]boxshadow.south east) rectangle
                ([xshift=\dimexpr\paperwidth-0.5cm\relax,yshift=-0.88cm]current page.north west);

            \node[anchor=west, align=left, text=white, font=\rmfamily\CJKfamily{XBS}\fontsize{7pt}{16pt}\selectfont]
            at ([xshift=0.1cm,yshift=-0.6cm]boxshadow.north east)
            {\spaceskip=0.5em plus 0.2em minus 0.1em 研究目的与现实意义};
            
            \begin{scope}[shift={($(current page.south)+(0cm,0.65cm)$)}]
                \path[rounded corners=0pt, left color=zjublue, right color=zjured, shading angle=90]
                    (-7.5cm,0) rectangle (7cm,0.5cm);            
            \end{scope}
            
            \node[anchor=south east, inner sep=0] at ([xshift=-0.2cm,yshift=0.3cm]current page.south east) 
                {\includegraphics[height=1.3cm]{images/logo_square.png}};

        
        \end{tikzpicture}
            
        \begin{tikzpicture}[remember picture, overlay]
            
            \node[anchor=north west, text width=15cm, align=justify, font=\rmfamily\CJKfamily{XBS}\fontsize{10pt}{10pt}\selectfont] at ([xshift=0.2cm,yshift=-1.2cm]current page.north west) 
            {
                \setbeamertemplate{itemize item}{\color{black}$\bullet$}
                \setbeamertemplate{itemize subitem}{\color{black}$\circ$}

                \begin{itemize}
                        
                    \item \textbf{标题一} \\[0.3em]
                    {\rmfamily\CJKfamily{simsun}\fontsize{8pt}{8pt} 这是标题一的正文,采用宋体书写,内容可以根据实际需要进行扩展和修改。内容可以根据实际需要进行扩展和修改。内容可以根据实际需要进行扩展和修改。内容可以根据实际需要进行扩展和修改。} \\                    
                    {\rmfamily\CJKfamily{kai}\fontsize{8pt}{8pt} 这里是标题一的补充说明,采用楷体书写,进一步阐述相关内容。} \\
                    [0.8em]
                                            
                \end{itemize}

                \begin{equation}
                    \mathnormal{F} = \mathnormal{G} \frac{\mathnormal{m}_{\mathnormal{1}} \mathnormal{m}_{\mathnormal{2}}}{\mathnormal{r}^{\mathnormal{2}}}
                \end{equation}
                            
                \hspace*{2.3em}\begin{minipage}{0.92\textwidth}
                    {\rmfamily\CJKfamily{simsun}\fontsize{8pt}{8pt} 该公式为万有引力定律,其中 $\mathnormal{F}$ 表示引力,$\mathnormal{m}_{\mathnormal{1}}$ 和 $\mathnormal{m}_{\mathnormal{2}}$ 分别为两个物体的质量,$\mathnormal{r}$ 为两物体间距离,$\mathnormal{G}$ 为引力常数。该公式描述了两个质量之间的引力与质量乘积成正比,与距离的平方成反比。}
                \end{minipage} \\
                
                \begin{equation}
                    \mathnormal{F} = \mathnormal{G} \frac{\mathnormal{m}_{\mathnormal{1}} \mathnormal{m}_{\mathnormal{2}}}{\mathnormal{r}^{\mathnormal{2}}}
                \end{equation}
                
                \hspace*{2.3em}\begin{minipage}{0.92\textwidth}
                    {\rmfamily\CJKfamily{simsun}\fontsize{8pt}{8pt} 该公式为万有引力定律,其中 $\mathnormal{F}$ 表示引力,$\mathnormal{m}_{\mathnormal{1}}$ 和 $\mathnormal{m}_{\mathnormal{2}}$ 分别为两个物体的质量,$\mathnormal{r}$ 为两物体间距离,$\mathnormal{G}$ 为引力常数。该公式描述了两个质量之间的引力与质量乘积成正比,与距离的平方成反比。} \\
                \end{minipage} \\
                

            };

            \node[align=center, font=\rmfamily\CJKfamily{XBS}\fontsize{10pt}{10pt}\selectfont, text=white] 
                at ([yshift=0.9cm]current page.south) {这里是渐变色矩形上的文本,可以根据需要进行修改。};
                
        \end{tikzpicture}

    \end{frame}


% 第四节
\section{请输入文本(结果与分析)}

    \begin{frame}{请输入文本(结果与分析)}

        %页面底色
        \begin{tikzpicture}[remember picture,overlay]
            
            \pgfmathsetlengthmacro{\rectheight}{\paperheight * 7 / 10}
                    
                    \fill[zjured] 
                        (current page.north west) -- 
                        (current page.north east) --
                        ($(current page.north east)-(0,\rectheight)$) --
                        ($(current page.north west)-(0,\rectheight)$) -- cycle;
                                        
                    \node[anchor=north east, inner sep=0] at 
                        ($(current page.north east)+(-0.5cm,-0.3cm)$) 
                        {\includegraphics[height=1.2cm]{images/logo.png}};

                    \begin{scope}[shift={($(current page.south west)+(0.5cm,0.5cm)$)}]
                        \path[rounded corners=0pt, left color=zjublue, right color=zjublue]
                            (0.7cm,0) rectangle (2cm,0.5cm);
                        \path[rounded corners=0pt, left color=zjublue, right color=zjured, shading angle=-135]
                            (0,0) rectangle (0.5cm,0.5cm);            
                    \end{scope}

                    \begin{scope}[shift={($(current page.south east)+(-0.5cm,0.5cm)$)}, xscale=-1]
                        \path[rounded corners=0pt, left color=zjublue, right color=zjublue]
                            (0.7cm,0) rectangle (2cm,0.5cm);
                        \path[rounded corners=0pt, left color=zjublue, right color=zjured, shading angle=135]
                            (0,0) rectangle (0.5cm,0.5cm);            
                    \end{scope}



        \end{tikzpicture}

        %页面文本
        \begin{tikzpicture}[remember picture, overlay]
        
            % 项目名称
            \node (activity_shadow) at ([xshift=1pt,yshift=-1pt,yshift=2.3cm]current page.center) {\centering\rmfamily\CJKfamily{simsun} \color{black!90}
            项目名称:请输入文本
            };
            \node (activity) at ([yshift=2.3cm]current page.center) {\centering\rmfamily\CJKfamily{simsun} \color{white} 
            项目名称:请输入文本
            };
            \draw[white, thin] (activity.west) -- (current page.west |- activity.west);
            \draw[white, thin] (activity.east) -- (current page.east |- activity.east);

            % 标题
            \node at ([xshift=2pt,yshift=-2pt,yshift=0.6cm]current page.center) {\centering\rmfamily\CJKfamily{BTS} \color{black!90} \fontsize{30pt}{5pt}\selectfont
            四、请输入文本(结果与分析)
            };
            \node at ([yshift=0.6cm]current page.center) {\centering\rmfamily\CJKfamily{BTS} \color{white} \fontsize{30pt}{5pt}\selectfont
            四、请输入文本(结果与分析)
            };
            \node at ([yshift=-0.7cm]current page.center) {\centering\rmfamily \color{white} \large 
            Subtitle: Please enter the text.
            };

            \node [text width=20cm] at ([yshift=-3cm]current page.center) {\centering\rmfamily\CJKfamily{kai} \fontsize{10pt}{15pt}\selectfont \setlength{\linespread{0.9}}
            小节内容1:请输入文本\\
            小节内容2:请输入\\
            小节内容3:文本Please\\
            };
            
        \end{tikzpicture}

    \end{frame}

    % 页面四
    \begin{frame}
    
        \begin{tikzpicture}[remember picture, overlay]

            \node[anchor=north west, inner sep=0pt] (boxshadow) at ([xshift=0.5cm,yshift=-0.5cm]current page.north west) {
                
                \begin{tikzpicture}                
                \node[fill=zjublue, text width=3cm, minimum height=0.8cm, 
                align=center,
                font=\rmfamily\CJKfamily{BTS}\fontsize{14pt}{14pt}\selectfont] 
                at (0,0) {
                    \begin{tikzpicture}[baseline=(textnode.base)]
                    \node[text=black!90, opacity=0.5, shift={(0.8pt,-0.5pt)}] (shadow) at (0,0) {\addfontfeature{LetterSpace=200}
                    研究意义};
                    \node[text=white] (textnode) at (0,0) {\addfontfeature{LetterSpace=200}
                    研究意义};
                    \end{tikzpicture}
                };
                \end{tikzpicture}

            };

            \draw[fill=zjublue, draw=none]
                ([xshift=0.1cm]boxshadow.north east) rectangle
                ([xshift=\dimexpr\paperwidth-0.5cm\relax,yshift=-0.8cm]current page.north west);

            \draw[fill=zjured, draw=none]
                ([xshift=0.1cm]boxshadow.south east) rectangle
                ([xshift=\dimexpr\paperwidth-0.5cm\relax,yshift=-0.88cm]current page.north west);

            \node[anchor=west, align=left, text=white, font=\rmfamily\CJKfamily{XBS}\fontsize{7pt}{16pt}\selectfont]
            at ([xshift=0.1cm,yshift=-0.6cm]boxshadow.north east)
            {\spaceskip=0.5em plus 0.2em minus 0.1em 研究目的与现实意义};
                
            \node[anchor=south east, inner sep=0] at ([xshift=-0.2cm,yshift=0.3cm]current page.south east) 
                {\includegraphics[height=1.3cm]{images/logo_square.png}};

        
        \end{tikzpicture}

        \begin{tikzpicture}[remember picture, overlay]
                
            \node[anchor=north east, text width=4cm, align=justify, font=\rmfamily\CJKfamily{XBS}\fontsize{10pt}{5pt}\selectfont] 
                        at ([xshift=-0.5cm,yshift=-1.8cm]current page.north east) 
                    {
                        
                        {\centering \textcolor{zjublue!50!zjured}{这里是右侧文本框标题}\\[1em]}
                        
                        {\rmfamily\CJKfamily{simsun}\fontsize{8pt}{8pt} 这是右侧文本框的正文,采用宋体书写,内容可以根据实际需要进行扩展和修改。} \\
                        {\rmfamily\CJKfamily{kai}\fontsize{8pt}{8pt} 这里是右侧文本框的补充说明,采用楷体书写,进一步阐述相关内容。} \\
                
                    };

                \node[anchor=north east, text width=4.3cm, align=justify, font=\rmfamily\CJKfamily{XBS}\fontsize{10pt}{5pt}\selectfont] 
                    at ([xshift=-0.5cm,yshift=-4.5cm]current page.north east) 
                {
                                
                    \setbeamertemplate{itemize item}{\color{black}$\circ$}
                    \begin{itemize}
                        \item {\rmfamily\CJKfamily{simsun}\fontsize{8pt}{8pt} 这是右侧文本框的正文,采用宋体书写,内容可以根据实际需要进行扩展和修改。} \\
                        \item {\rmfamily\CJKfamily{kai}\fontsize{8pt}{8pt} 这里是右侧文本框的补充说明,采用楷体书写,进一步阐述相关内容。} \\
                    \end{itemize}
                    
                };
                    
            % 表格
            \node[anchor=north west, text width=10.2cm, align=justify] at ([xshift=0.2cm,yshift=-2.2cm]current page.north west) {
                \begin{table}[h]
                    \centering
                    {\rmfamily\CJKfamily{simsun}\fontsize{8pt}{8pt}\selectfont
                    \begin{tabular}{lccc}
                        \toprule[1pt]
                        \multicolumn{1}{l}{\textbf{指标}} & 
                        \multicolumn{1}{c}{\textbf{均值}} & 
                        \multicolumn{1}{c}{\textbf{标准差}} & 
                        \multicolumn{1}{c}{\textbf{样本量}} \\ 
                        \midrule[0.5pt]
                        人均GDP(万元)    & 6.82  & 2.43 & 120 \\[0.3em]
                        固定资产投资(亿元)& 892.5 & 456.2 & 120 \\[0.3em]
                        居民消费指数       & 103.2 & 1.86 & 120 \\[0.3em]
                        工业增加值增速(\%)& 7.23  & 2.15 & 120 \\[0.3em]
                        社会消费品零售总额  & 436.8 & 278.4 & 120 \\
                        \bottomrule[1pt]
                    \end{tabular}
                    }
                    \vspace{0.5em}
                    \\
                    {\rmfamily\CJKfamily{kai}\fontsize{8pt}{8pt}\selectfont \raggedright
                    注:数据来源于国家统计局,样本区间为2010-2019年。}              
                \end{table}
            };
        
    
        \end{tikzpicture}


    \end{frame}

% 第五节
\section{请输入文本(总结与讨论)}

    \begin{frame}{请输入文本(总结与讨论)}

        %页面底色
        \begin{tikzpicture}[remember picture,overlay]
            
            \pgfmathsetlengthmacro{\rectheight}{\paperheight * 7 / 10}
                    
                    \fill[zjublue] 
                        (current page.north west) -- 
                        (current page.north east) --
                        ($(current page.north east)-(0,\rectheight)$) --
                        ($(current page.north west)-(0,\rectheight)$) -- cycle;
                                        
                    \node[anchor=north east, inner sep=0] at 
                        ($(current page.north east)+(-0.5cm,-0.3cm)$) 
                        {\includegraphics[height=1.2cm]{images/logo.png}};

                    \begin{scope}[shift={($(current page.south west)+(0.5cm,0.5cm)$)}]
                        \path[rounded corners=0pt, left color=zjured, right color=zjured]
                            (0.7cm,0) rectangle (2cm,0.5cm);
                        \path[rounded corners=0pt, left color=zjured, right color=zjublue, shading angle=-135]
                            (0,0) rectangle (0.5cm,0.5cm);            
                    \end{scope}

                    \begin{scope}[shift={($(current page.south east)+(-0.5cm,0.5cm)$)}, xscale=-1]
                        \path[rounded corners=0pt, left color=zjured, right color=zjured]
                            (0.7cm,0) rectangle (2cm,0.5cm);
                        \path[rounded corners=0pt, left color=zjured, right color=zjublue, shading angle=135]
                            (0,0) rectangle (0.5cm,0.5cm);            
                    \end{scope}



        \end{tikzpicture}

        %页面文本
        \begin{tikzpicture}[remember picture, overlay]
        
            % 项目名称
            \node (activity_shadow) at ([xshift=1pt,yshift=-1pt,yshift=2.3cm]current page.center) {\centering\rmfamily\CJKfamily{simsun} \color{black!90}
            项目名称:请输入文本
            };
            \node (activity) at ([yshift=2.3cm]current page.center) {\centering\rmfamily\CJKfamily{simsun} \color{white} 
            项目名称:请输入文本
            };
            \draw[white, thin] (activity.west) -- (current page.west |- activity.west);
            \draw[white, thin] (activity.east) -- (current page.east |- activity.east);

            % 标题
            \node at ([xshift=2pt,yshift=-2pt,yshift=0.6cm]current page.center) {\centering\rmfamily\CJKfamily{BTS} \color{black!90} \fontsize{30pt}{5pt}\selectfont
            五、请输入文本(总结与讨论)
            };
            \node at ([yshift=0.6cm]current page.center) {\centering\rmfamily\CJKfamily{BTS} \color{white} \fontsize{30pt}{5pt}\selectfont
            五、请输入文本(总结与讨论)
            };
            \node at ([yshift=-0.7cm]current page.center) {\centering\rmfamily \color{white} \large 
            Subtitle: Please enter the text.
            };

            \node [text width=20cm] at ([yshift=-3cm]current page.center) {\centering\rmfamily\CJKfamily{kai} \fontsize{10pt}{15pt}\selectfont \setlength{\linespread{0.9}}
            小节内容1:请输入文本\\
            小节内容2:请输入\\
            小节内容3:文本Please\\
            };
            
        \end{tikzpicture}

    \end{frame}

    % 页面五
    \begin{frame}
    
        \begin{tikzpicture}[remember picture, overlay]

            \node[anchor=north west, inner sep=0pt] (boxshadow) at ([xshift=0.5cm,yshift=-0.5cm]current page.north west) {
                
                \begin{tikzpicture}                
                \node[fill=zjublue, text width=3cm, minimum height=0.8cm, 
                align=center,
                font=\rmfamily\CJKfamily{BTS}\fontsize{14pt}{14pt}\selectfont] 
                at (0,0) {
                    \begin{tikzpicture}[baseline=(textnode.base)]
                    \node[text=black!90, opacity=0.5, shift={(0.8pt,-0.5pt)}] (shadow) at (0,0) {\addfontfeature{LetterSpace=200}
                    研究意义};
                    \node[text=white] (textnode) at (0,0) {\addfontfeature{LetterSpace=200}
                    研究意义};
                    \end{tikzpicture}
                };
                \end{tikzpicture}

            };

            \draw[fill=zjublue, draw=none]
                ([xshift=0.1cm]boxshadow.north east) rectangle
                ([xshift=\dimexpr\paperwidth-0.5cm\relax,yshift=-0.8cm]current page.north west);

            \draw[fill=zjured, draw=none]
                ([xshift=0.1cm]boxshadow.south east) rectangle
                ([xshift=\dimexpr\paperwidth-0.5cm\relax,yshift=-0.88cm]current page.north west);

            \node[anchor=west, align=left, text=white, font=\rmfamily\CJKfamily{XBS}\fontsize{7pt}{16pt}\selectfont]
            at ([xshift=0.1cm,yshift=-0.6cm]boxshadow.north east)
            {\spaceskip=0.5em plus 0.2em minus 0.1em 研究目的与现实意义};
                
            \node[anchor=south east, inner sep=0] at ([xshift=-0.2cm,yshift=0.3cm]current page.south east) 
                {\includegraphics[height=1.3cm]{images/logo_square.png}};

        
        \end{tikzpicture}

        \begin{tikzpicture}[remember picture, overlay]
            
            \node[anchor=north west, text width=15cm, align=justify, font=\rmfamily\CJKfamily{XBS}\fontsize{10pt}{10pt}\selectfont] at ([xshift=0.2cm,yshift=-1.2cm]current page.north west) 
            {
                
                \setbeamertemplate{itemize item}{\color{black}$\bullet$}
                \setbeamertemplate{itemize subitem}{\color{black}$\circ$}

                \begin{itemize}
                    \item \textbf{标题一} \\[0.3em]
                    {\rmfamily\CJKfamily{simsun}\fontsize{8pt}{8pt} 这是标题一的正文,采用宋体书写,内容可以根据实际需要进行扩展和修改。内容可以根据实际需要进行扩展和修改。内容可以根据实际需要进行扩展和修改。内容可以根据实际需要进行扩展和修改。} \\                    
                    {\rmfamily\CJKfamily{kai}\fontsize{8pt}{8pt} 这里是标题一的补充说明,采用楷体书写,进一步阐述相关内容。} \\
                \end{itemize}
                \vspace{-2em} % 非必需
                \begin{center}
                    \begin{minipage}{0.35\textwidth}
                        \centering
                        \includegraphics[height=5cm]{images/logo_square.png}\\[0.5em]
                        {\vspace{-1.5em}\rmfamily\CJKfamily{kai}\fontsize{8pt}{8pt} 这是左侧图片的说明文字,采用楷体,内容可根据实际需要进行修改。}
                    \end{minipage}
                    \hspace{2cm}
                    \begin{minipage}{0.35\textwidth}
                        \centering
                        \includegraphics[height=5cm]{images/logo_square.png}\\[0.5em]
                        {\vspace{-1.5em}\rmfamily\CJKfamily{kai}\fontsize{8pt}{8pt} 这是右侧图片的说明文字,采用楷体,内容可根据实际需要进行修改。}
                    \end{minipage}
                \end{center}
                            
            };


                
        \end{tikzpicture}


    \end{frame}

% 参考文献
\begin{frame}[plain]

    % 页面排版
    \begin{tikzpicture}[remember picture, overlay]

        \begin{scope}[scale=0.8]

            \node[anchor=center, inner sep=0, opacity=0.1] at 
                (current page.center)
                {\includegraphics[height=11cm]{images/logo_square.png}};

            % 渐变矩形背景
            \path[rounded corners=0pt, left color=zjublue, right color=zjured, shading angle=-45, opacity=1]
                ([xshift=-1.8cm,yshift=-1.8cm]current page.north) rectangle ([xshift=1.8cm,yshift=-0.8cm]current page.north);

            % 参考文献文本
            \node at ([xshift=1pt,yshift=-1.3cm,yshift=-1pt]current page.north) {\rmfamily\CJKfamily{BTS} \fontsize{14pt}{5pt}\selectfont \color{black!80}
                \kern0.2em 
                参\hspace{5pt}考\hspace{5pt}文\hspace{5pt}献
                \kern0.2em
                };
                \node at ([yshift=-1.3cm]current page.north) {\rmfamily\CJKfamily{BTS} \fontsize{14pt}{5pt}\selectfont \color{white}
                \kern0.2em 
                参\hspace{5pt}考\hspace{5pt}文\hspace{5pt}献
                \kern0.2em
            };

        \end{scope}

    \end{tikzpicture}

    \printbibliography


\end{frame}



% 结束页
\begin{frame}[plain]
    
    % 封面底色形状
    \begin{tikzpicture}[remember picture,overlay]
        
        \pgfmathsetlengthmacro{\rectheight}{\paperheight * 7 / 10}
        
        \fill[zjublue] 
            (current page.north east) -- 
            (current page.north west) --
            ($(current page.north west)-(0,\rectheight)$) --
            ($(current page.north east)-(0,\rectheight)$) -- cycle;
        
        \fill[zjured] 
            (current page.north east) -- 
            ($(current page.north west)-(0,\rectheight)$) --
            ($(current page.north east)-(0,\rectheight)$) --
            (current page.north east) -- cycle;

        \node[anchor=north west, inner sep=0] at 
            ($(current page.north west)+(0.5cm,-0.3cm)$) 
            {\includegraphics[height=1.2cm]{images/logo.png}};
       
        \begin{scope}[shift={($(current page.south west)+(0.5cm,1.75cm)$)}, scale=0.8, yscale=-1]
            \path[fill=none, draw=none, name path=leftL]
            (0,0) -- ++(0,1.5cm) -- ++(1.5cm,0) -- ++(0,-0.5cm) -- ++(-1cm,0) -- ++(0,-1cm) -- cycle;
            \path[left color=zjublue, right color=zjured, shading angle=-135, opacity=0.9]
            (0,0) -- ++(0,1.5cm) -- ++(1.5cm,0) -- ++(0,-0.5cm) -- ++(-1cm,0) -- ++(0,-1cm) -- cycle;
        \end{scope}

        \begin{scope}[shift={($(current page.south east)+(-0.5cm,1.75cm)$)}, scale=0.8]
            \path[fill=none, draw=none, name path=rightL]
                (0,0) -- ++(0,-1.5cm) -- ++(-1.5cm,0) -- ++(0,0.5cm) -- ++(1cm,0) -- ++(0,1cm) -- cycle;
            \path[left color=zjured, right color=zjublue, shading angle=135, opacity=0.9]
                (0,0) -- ++(0,-1.5cm) -- ++(-1.5cm,0) -- ++(0,0.5cm) -- ++(1cm,0) -- ++(0,1cm) -- cycle;
        \end{scope}


    \end{tikzpicture}

    % 封面文本
    \begin{tikzpicture}[remember picture, overlay]
       
        % 活动名称
        \node (activity_shadow) at ([xshift=1pt,yshift=-1pt,yshift=2.3cm]current page.center) {\rmfamily\CJKfamily{simsun} \color{black!90}
        活动名称:请输入文本
        };
        \node (activity) at ([yshift=2.3cm]current page.center) {\rmfamily\CJKfamily{simsun} \color{white} 
        活动名称:请输入文本
        };
        \draw[white, thin] (activity.west) -- (current page.west |- activity.west);
        \draw[white, thin] (activity.east) -- (current page.east |- activity.east);

        % 标题
        \node at ([xshift=2pt,yshift=-2pt,yshift=0.6cm]current page.center) {\rmfamily\CJKfamily{BTS} \color{black!90} \fontsize{40pt}{5pt}\selectfont
        汇报结束 \hspace{0.1pt} 敬请指正
        };
        \node at ([yshift=0.6cm]current page.center) {\rmfamily\CJKfamily{BTS} \color{white} \fontsize{40pt}{5pt}\selectfont
        汇报结束 \hspace{0.1pt} 敬请指正
        };
        \node at ([yshift=-0.6cm]current page.center) {\rmfamily \color{white} \large 
        Thank You For Listening.
        };

        % 指导教师
        \node at ([yshift=-2.3cm]current page.center) {\rmfamily\CJKfamily{kai} \fontsize{10pt}{5pt}\selectfont
        指导教师:请输入文本
        };
        % 所属单位
        \node at ([yshift=-2.8cm]current page.center) {\rmfamily\CJKfamily{kai} \fontsize{10pt}{5pt}\selectfont
        所属单位:请输入文本
        };
        % 姓名
        \node at ([yshift=-3.3cm]current page.center) {\rmfamily\CJKfamily{kai} \fontsize{10pt}{5pt}\selectfont
        姓名:请输入文本
        };
        % 日期
        \node at ([yshift=-3.8cm]current page.center) {\rmfamily\CJKfamily{kai} \fontsize{10pt}{5pt}\selectfont
        日期:\today
        };
    
    \end{tikzpicture}

\end{frame}



\end{document}